\section{Offene Arbeiten}
Um die Arbeit erfolgreich und möglichst speditiv fortzuführen, werden in diesem Kapitel die noch offenen Arbeiten und weiteren Schritte erläutert.

Nach diversen erfolgreichen Tests am Motor konnten wir praktisch alle Anforderungen (Anhang \ref{appsec:Anforderungen}) an den Einzugswindenbetrieb erledigen. Da das Produkt aber sowohl einen Einzugs- als auch einen Abrollwindenbetrieb garantieren soll, muss in einem weiteren Schritt der Motor im Generatorbetrieb getestet werden. Damit dies erfolgreich erledigt werden kann, muss eine Bremsschaltung aufgebaut werden, welche die rekuperierte Energie des BLDC-Motors zum Aufladen der Batterien oder in Wärme umwandeln kann. Es muss beachtet werden, dass beim Abrollbetrieb Leistungen von bis zu 10kW entstehen können. Diese Leistung wird primär benutzt um die Batterien zu laden. Sobald diese vollständig geladen sind, wird die Energie über Heizwiderstände in Wärme umgewandelt.

Die Auswahl der Batterien ist aus diesem Grund nicht nur für die Energieversorgung beim Einzugswindenbetrieb, sondern auch für den Abrollwindenbetrieb von grosser Bedeutung. Aus diesem Grund ist auch diese Anschaffung ein weiterer Bestandteil, welcher als Nächstes ins Auge gefasst werden muss. Mit der Anschaffung von Batterien ist das ganze Thema der Energieversorgung jedoch noch nicht erledigt. Es ist ein Ladegerät notwendig, welches ein Ladevorgang während dem Einzugswindenbetrieb ermöglicht. Zudem ist auch eine Batterieüberwachung notwendig, welche die Batterie gegen Über-, und Tiefentladung schützt und den Windenführer vorzeitig warnt. Wie diese Warnung erreicht wird, fällt in die Ausarbeitung der Bedieneinheit, welche ebenfalls ausgearbeitet werden muss. Die Ansteuerung des Motors mit den unterschiedlichen Start Abläufen, die Bedienung des Not-Aus, eine mögliche visuelle Anzeige über aktuelle Parameter und die Positionierung am Endgerät, müssen zudem ausgearbeitet werden.

Zum Schutz vor Berührung, werden alle elektronischen Komponenten (wie Controller, Chopper-Schaltung, usw.) in einen Schaltschrank eingebaut, welcher ebenfalls eine Sicherung über das Gesamtsystem enthält.
Weiter ist gemäss DHV-Reglement eine Rundumleuchte zur Betriebssignalisation erforderlich.\\
Zum Schluss stehen noch zahlreiche Tests bevor. Die Anlage wird zuerst im und ausserhalb des Labors unter idealen Bedingungen getestet, bevor der erste Gleitschirmflieger in einem Feldversuch hochgezogen wird.