\section{Offene Arbeiten}
Um die Arbeit erfolgreich und möglichst speditiv fortzuführen, werden in diesem Kapitel die noch offenen Arbeiten und die weiteren Schritte, welche verfolgt werden erläutert.

Nach diversen erfolgreichen Tests am Motor konnten wir praktisch alle Anforderungen (Anhang \ref{appsec:Anforderungen}) an den Einzugswindenbetrieb erledigen. Da das Produkt aber sowohl einen Einzugs- als auch einen Abrollwindenbetrieb garantieren soll, muss in einem weiteren Schritt der Motor im Betrieb als Generator getestet werden. Damit dies erfolgreich erledigt werden kann, muss eine Bremsschaltung aufgebaut werden, welche die rekuperierte Energie des BLDC-Motors in Wärme umwandeln kann. Beim Abrollbetrieb können Leistungen von bis zu 10kW entstehen. Diese Leistung wird primär benutzt um die Batterien zu laden, sobald diese jedoch geladen sind, muss diese Energie verheizt werden. Diese Heizwiderstände und die dazugehörige Schaltung muss dafür dimensioniert werden.

Die Auswahl der Batterie ist aus diesem Grund nicht nur für die Energieversorgung des Motors von grosser Bedeutung, sondern auch für den Abrollbetrieb. Aus diesem Grund ist auch diese Anschaffung ein weiterer Bestandteil, welcher als Nächstes ins Auge gefasst werden muss. Mit der Anschaffung von Batterien ist das ganze Thema der Energieversorgung jedoch noch nicht erledigt. Es ist ein Ladegerät notwendig, sofern die Winde nur als Einzugswinde betrieben wird. Zudem ist auch eine Batterieüberwachung notwendig, welche die Batterie gegen Über-, und Tiefenentladung schützt und vorzeitig warnt. Wie diese Warnung erreicht wird, fällt in die Ausarbeitung der Bedieneinheit, welche ebenfalls ausgearbeitet werden muss. Die Ansteuerung des Motors mit den unterschiedlichen Start Abläufen, die Bedienung des Not-Aus, eine mögliche visuelle Anzeige über aktuelle Parameter und die Positionierung stehen zudem im Fokus.

Zum Schutz vor Berührung, werden alle elektronischen Komponenten (wie Controller, Chopper-Schaltung, usw.) in einen Schaltschrank eingebaut, welcher ebenfalls eine Sicherung über das Gesamtsystem enthält. Die damit verbundene Übersichtlichkeit ist zudem wartungsfreundlich und ermöglicht rasche Anpassungen. Weiter ist gemäss DHV-Reglement eine Rundumleuchte zur Betriebssignalisation erforderlich.\\
Zum Schluss stehen noch zahlreiche Tests bevor. Die Anlage wird zuerst im und ausserhalb des Labors unter idealen Bedingungen getestet, bevor der erste Gleitschirmflieger in einem Feldversuch hochgezogen wird.