\section{Offene Arbeiten}
Um die Arbeit erfolgreich und möglichst speditiv fortzuführen, werden in diesem Kapitel die noch offenen Arbeiten und weiteren Schritte erläutert.\\

Nach diversen erfolgreichen Tests konnte bewiesen werden, dass der Motor alle Anforderungen an den Einzugswindenbetrieb erfüllen kann.\\
Da das Produkt sowohl einen Einzugswinden- als auch einen Abrollwindenbetrieb garantieren soll, muss in einem weiteren Schritt der Motor im Generatorbetrieb getestet werden. Die erzeugte Energie wird in diesem Betriebsmodus primär genutzt, um die Batterien aufzuladen. Sobald diese vollständig geladen sind, werden Heizwiderstände dazugeschaltet um eine Überladung der Batterien zu verhindern. Dieser Vorgang wird durch eine Entladeschaltung mit zugehöriger Elektronik dimensioniert und getestet. Um diese Tests durchführen zu können, sind somit Batterien zwingend notwendig und müssen zu Beginn des fortführenden Projekts bestellt werden.\\
Weiter ist ein Ladegerät notwendig, welches einen Ladevorgang während dem Einzugswindenbetrieb ermöglicht. Dies lässt selbst bei erschöpften Batterien einen Schleppbetrieb zu.\\
Zudem ist auch eine Batterieüberwachung notwendig, welche die Batterie gegen Über-, und Tiefentladung schützt und den Windenführer vorzeitig warnt. Wie diese Warnung erreicht wird, fällt in die Ausarbeitung der Bedieneinheit. Sie beinhaltet zudem die Ansteuerung des Motors mit den unterschiedlichen Start Abläufen, die Bedienung des Not-Aus, sowie eine visuelle Anzeige über aktuelle Parameter.\\
Zum Schutz vor Berührung, werden alle elektronischen Komponenten (wie Controller, Chopper-Schaltung, usw.) in einen Schaltschrank eingebaut, welcher ebenfalls eine Sicherung über das Gesamtsystem enthält.\\
Weiter ist gemäss DHV-Reglement eine Rundumleuchte zur Betriebssignalisation erforderlich.\\
Zum Schluss stehen noch zahlreiche Tests bevor. Die Anlage wird zuerst im und ausserhalb des Labors getestet, bevor der erste Gleitschirmpilot hochgezogen wird.