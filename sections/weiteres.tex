\section{Weitere Arbeiten}
Um die Arbeit erfolgreicht und möglichst speditiv fortzuführen, werden in diesem Kapitel die noch offenen Arbeiten und die weiteren Schritte welche verfolgt werden erläutert.

Nach diversen erfolgreichen Tests am Motor, konnten wir praktisch alle Anforderungen an den Einzugswindenbetrieb erledigen. Da unser Produkt aber sowohl einen Einzugs-, als auch einen Abrollwindenbetrieb garantieren soll, muss in einem weiteren Schritt der Motor im Betrieb als Generator getestet werden. Damit dies erfolgreich erledigt werden kann, muss eine Bremsschaltung aufgebaut werden, welche die rekuperierte Energie des BLDC Motors in Wärme umwandeln kann. Da Leistungen von bis zu 10kW entstehen können, und einzelne Widerstände mit einer solch hohen Leistung extrem rar und dazu sehr teuer sind, werden vorzugsweise mehrere einzelne Widerstände seriell geschaltet. Zudem ist die Idee, dass ein Teil der Energie für den Ladezyklus der Batterien verwendet werden kann.
Die Auswahl der Batterie ist aus diesem Grund nicht nur für die Energieversorgung des Motors von grosser Bedeutung, sondern auch für den Abrollbetrieb. Aus diesem Grund ist auch diese Anschaffung ein weiterer Bestandteil welcher als nächstes ins Auge gefasst werden muss. Mit der Anschaffung von Batterien ist das ganze Thema der Energieversorgung jedoch noch nicht erledigt. Es ist ein Ladegerät notwendig, sofern die Winde nur als Einzugswinde betrieben wird. Zudem ist auch eine Batterieüberwachung notwendig, welche die Batterie gegen Über-, und Tiefenentladung schützt und vorzeitig warnt. Wie diese Warnung erreicht wird, fällt in die Ausarbeitung der Bedieneinheit, welche ebenfalls ausgearbeitet werden muss. Die Ansteuerung des Motors mit den unterschiedlichen Start Abläufen, die Bedienung des Not-Aus, eine mögliche visuelle Anzeige über aktuelle Parameter und die Positionierung stehen zudem im Fokus.
Zu guter Letzt ist auch die Sicherheit eines der wichtigsten Elemente welche in jedem Fall gewährleistet werden muss. So muss einerseits eine gelbe Rundumleuchte installiert werden, aber auch die gesamte Elektronik bei fehlerfällen durch eine Sicherung geschützt werden. Die Positionierung der Sicherung und anderen bereits erwähnten Elektronik Komponenten (wie Controller und Chopper Schaltung) werden  in einem Elektronikkasten untergebracht.
Abgeschlossen wird das Projekt mit einem ausführlichen Test, welcher auf dem Feld durchgeführt wird.