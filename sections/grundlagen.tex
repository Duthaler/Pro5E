\section{Grundlagen}\label{sec:Grundlagen}
In diesem Kapitel werden für den Bericht grundlegende Informationen beschrieben. Diese werden zum einen in Richtlinien unterteilt,welche hauptsächlich aus den gesetzlichen Bestimmungen bestehen. Zum anderen ist ein mathematischer Teil vorhanden, wobei die Berechnungen mathematische Formeln aufweisen auf welche in den nachfolgenden Kapiteln verwiesen wird. 

\subsection{Richtlinien}\label{subsec:Richtlinien}
Die gesetzlichen Bestimmungen für die gesamte Seilwinde wurden bereits im Pflichtenheft thematisiert und aufgeschrieben. Nachfolgend werden daher lediglich die für das Testing des Motors und Controllers relevanten Vorschriften aufgeführt. Für eine detaillierte Übersicht aller Anforderungen wird auf das technische Pflichtenheft verwiesen \cite{TechPflichtenheft}.

Die Zugkraft der Seilwinde (respektive des Seils) darf beim Einzelschlepp den Wert von $ 1000N $ nicht überschreiten. Werden zwei Personen gleichzeitig geschleppt (in Form eines Tandemschlepps), darf diese auf maximal $ 1300N $ erhöht werden. Weiter darf während dem Schleppvorgang eine maximale Welligkeit von $\pm 25N$ auftreten. Das bedeutet, dass ein Oszillierender Einzug mit der genannten Kraftamplitude erlaubt ist. Wird der Winde ein Drehmomentsollwert vorgegeben, so darf diese nicht mehr als $\pm 100N$ davon abweichen. Das bedeutet bei maximaler Zugkraft (im Einzelschlepp) dürfen Werte zwischen $ 900N $ und $ 1100N $ auftreten. Aus Sicherheitsgründen wird die maximale Einzugsgeschwindigkeit begrenzt. Da bei Nullwind eine Einzugsgeschwindigkeit von 10m/s für opitmales Steigen sorgt und evt. leichte Rückenwinde während dem Schleppvorgang auftreten können, beträgt die Geschwindigkeit des Seil maximal auf $ 12m/s $.



\subsection{Berechnungen}\label{subsec:Berechnungen}


\begin{table}[H]
	\centering
	\begin{tabular}{C{5cm} C{2.5cm}}
		\multicolumn{2}{c}{}\\
		{Beschreibung} & {Index} \\ \hline
		Zwischenkreisspannung    &   $  U_{di0} $   \\
		Verkettete Spannung    &   $ U_{AD} $   \\
		Zwischenkreisstrom    &   $ I_{d} $  \\
		Leiterstrom AC-Seite    &  $ I_{A,rms} $   \\
		Cracking Amperes    &  $ CA $  \\
		Kapazität Batterie    &  $ Q $   \\
	\end{tabular}
	\caption{Index Erläuterungen für Berechnungen}
	\label{tab:Begriffserklaerung}
\end{table}



\textbf{B6-Gleichrichter Zwischenkreisspannung}
\begin{equation}
	\centering
	U_{di0}=\frac{3}{\pi}\sqrt{2}\cdot U_{AD}=1.35U_{AD}
	\label{eq:B6-Ud}
\end{equation}

\textbf{B6-Gleichrichter Zwischenkreisstrom}
\begin{equation}
	\centering
	I_{d}=\frac{I_{A,rms}}{\sqrt{\frac{2}{3}}}
	\label{eq:B6-Id}
\end{equation}

\textbf{Batterie Lade-, Entladestrom mittels CA-Wert}
\begin{equation}
	\centering
	I=CA \cdot Q
	\label{eq:CA}
\end{equation}

\begin{equation}
\centering
P = M \cdot \omega = M \cdot 2 \cdot \pi\cdot f
\label{eq:Leistung}
\end{equation}