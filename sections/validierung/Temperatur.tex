\subsection{Temperatur}\label{subsec:Temperatur}
Um einen Einblick in das Temperaturverhalten des Motors zu erlangen, wurde in diesem Versuch die Erwärmung in Abhängigkeit der Drehzahl bei konstantem Moment gemessen. Die Messung erfolgte bei einer Drehzahl von 1550 RPM und bei 2550 RPM, und einem jeweiligen Drehmoment von 32Nm. Die Starttemperatur des Motors lag jeweils bei ca. 60°C und wurde solang betrieben, bis er eine Betriebstemperatur von 100°C erreicht hatte.


\begin{figure}[H]
	\centering
	\includegraphics[width=0.8\linewidth]{Temperatur.jpg}
	\caption{Erwärmung}\label{fig:Temperatur}
\end{figure}


Da die Leistung sowohl vom Drehmoment als auch von der Drehzahl abhängig ist, wurde beim Versuch bei 2550RPM eine höhere Leistung und somit auch ein grösserer Strom erzielt. Aus diesem Grund ist die Erwärmung des Motors bei höheren Drehzahlen dementsprechend grösser. Anhand der beiden Versuchen gehen wir davon aus, dass der BLDC-Motor ca. 5 Minuten unter Vollast betrieben werden kann, bis er die 100°C erreicht. Es gilt zu beachten, dass der Motor eine Betriebstemperatur von 110°C zulässt, wodurch eine Reserve von 10°C gegeben ist.