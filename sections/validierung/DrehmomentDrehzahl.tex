\subsection{Drehmoment bei variabler Drehzahl}\label{subsec:DrehmomentDrehzahl}
Um zu bewiesen, dass der Motor die erforderliche Leistung erbringen kann, wurde das Drehmoment in Abhängigkeit der Drehzahl untersucht.

% dieser Teil eventulle in den Hartwareteil verschieben und nur darauf Referenzieren
Bei einem minimalen mechanischen Wirkungsgrad von 0.8, einem 1:16 Getriebe und einem Trommeldurchmesser von 80cm, muss der Motor ein konstantes Drehmoment von 32Nm liefern. 

Die Bedingungen, mit welcher der Versuch durchgeführt wurde, können der Tabelle \ref{tab:Drehmoment/Drehzahl} entnommen werden.

\begin{table}[H]
\centering
\begin{tabular}{C{4cm} C{4cm} C{3cm}} 
\multicolumn{3}{c}{\textbf{Versuchsbedingungen}} \\
{Messgrösse}& {Bedingung} & {Wert}\\ \hline\hline 
Spannung (DC)   & nachgeregelt &   96 V     \\
Strom (DC)   & gemessen &   37.8-128 A     \\
Leistung (AC)   & gemessen &   1702-8870 W    \\
Drehzahl   & variiert &   614-2954 RPM    \\
Drehmoment-Sollwert   & nachgeregelt &   32 Nm    \\
Motor-Temperatur   & vernachlässigt &   -    \\
Controller-Temperatur   & vernachlässigt &   -    \\
\end{tabular}
\caption{Versuchsbedingungen Drehmoment/Drehzahl-Versuch}\label{tab:Drehmoment/Drehzahl}
\end{table}

Das Drehmoment an der Welle des BLDC-Motors wird mithilfe der Formel (FORMEL IN HARWARE/GRUNDLAGEN) über die Leistung und der Drehzahl des Motors ermittelt. Diese Kurve ist in der Abbildung \ref{fig:drehmoment/drehzahl} (blaue Kurve) ersichtlich. Da die asynchrone Maschine ebenfalls nicht ideal arbeitet, wird bei dieser einen Wirkungsgrad von 90\% angenommen (rote Kurve).

\begin{figure}[H]
	\centering
	\includegraphics[width=0.8\linewidth]{drehmoment_drehzahl.jpg}
	\caption{Drehmoment in Abhängigkeit der Drehzahl}\label{fig:drehmoment/drehzahl}
\end{figure}

Bei diesem Versuch ist ersichtlich, dass es möglich war, die erforderliche Leistung (schwarze Punkte) im Drehzahlbereich zwischen 600 und 3000 RPM zu erreichen. Die aufgenommene Leistung auf der DC-Seite (violette Punkte) ist dabei proportional zur abgegebenen Leistung angestiegen und erreicht bei 3000 RPM einen Wert von über 12kW.

In der Abbildung \ref{fig:drehmoment/StromSpannung} sind die Spannung, der Strom und die Sollwertvorgabe für die Ansteuerung während des Versuchs ersichtlich. Die Spannung wurde auf $96V_{DC}$ nachgeregelt (blaue Punkte), damit diese für den Versuch konstant blieb. Der Strom (rote Punkte) und der Sollwert für die Ansteuerung (orange Punkte) wurden ebenfalls während des Versuchs dokumentiert.


\begin{figure}[H]
	\centering
	\includegraphics[width=0.8\linewidth]{drehmoment_StromSpannung.jpg}
	\caption{Spannung und Strom während des Drehmomentversuchs}\label{fig:drehmoment/StromSpannung}
\end{figure}

%Welcher graue Transformator? Verweis auf ein Versuchsaufbau Bild?
Da der graue Transformator nur für einen Strom von 95A konzipiert ist, was bei der verwendeten B6-Brückenschaltung einem Strom (gemäss Formel \ref{eq:B6}) von rund 128A im DC-Zwischenkreis entspricht, konnte der Versuch nicht bis 3800 RPM durchgeführt werden. Der Strom kann mit guter Näherung als linear zur Drehzahl bei gleichbleibendem Drehmoment betrachtet werden. Dadurch ist ersichtlich, dass bei 3800 RPM ein Strom von ca. 160A benötigt wird. Dies Stromstärke deckt sich auch mit dem Wert im Datenblatt des Motors (REFERENZ AUF DATENBLATT).

Bei diesem Versuch ist ebenfalls ersichtlich, dass die Sollwertvorgabe über den Drehzahlbereich nur kleine Abweichungen hat und daher in guter Näherung als konstant betrachtet werden kann.