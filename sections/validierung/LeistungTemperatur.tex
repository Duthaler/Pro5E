\subsection{Leistungsabhängigkeit der Temperatur}\label{subsec:Leistung/Temperatur}
Da der BLDC-Motor eine sehr kleine Bauform für seine Grösse hat, unterliegt dieser grossen Temperaturschwankungen. Hierbei wird untersucht, wie sich die Temperatur bei konstantem Sollwert auf die Leistung auswirkt. Die Versuchsbedingungen können der Tabelle \ref{tab:Leistung/Temperatur} entnommen werden.


\begin{table}[H]
	\centering
	\begin{tabular}{C{4cm} C{4cm} C{3cm}} 
		\multicolumn{3}{c}{\textbf{Versuchsbedingungen}} \\
		{Messgrösse}& {Bedingung} & {Wert}\\ \hline\hline 
		Spannung (DC)   & nachgeregelt &   96 V     \\
		Strom (DC)   & gemessen &   106-112 A     \\
		Leistung (AC)   & gemessen &   7330-7820 W    \\
		Drehzahl   & konstant &   2500 RPM    \\
		Drehmoment-Sollwert   & konstant &   32 Nm    \\
		Motor-Temperatur   & gemessen &   30-100 °C    \\
		Controller-Temperatur   & vernachlässigt &   -    \\
	\end{tabular}
	\caption{Versuchsbedingungen Leistung/Temperatur-Versuch}\label{tab:Leistung/Temperatur}
\end{table}

Wie in der Abbildung \ref{fig:Leistung/Temperatur} ersichtlich ist, nimmt die Leistung bei zunehmender Temperatur ab. Die Leistung verringerte sich bei einer Motor-Temperatur von 100°C um ca. 6\% gegenüber einer Temperatur von 30°C. Die Leistungsabnahme ist dabei in guter Näherung linear zur Temperatur.

\begin{figure}[H]
	\centering
	\includegraphics[width=0.8\linewidth]{LeistungTemperatur.jpg}
	\caption{Einfluss der Erwärmung auf die Leistung}\label{fig:Leistung/Temperatur}
\end{figure}

Wird in einem der nachfolgenden Versuche erwähnt, dass die Leistungs-Temperaturabhängigkeit berücksichtigt wurde, denn werden die gegeben Messwerte mit der linearen Approximation auf eine Betriebstemperatur von 70°C berechnet.