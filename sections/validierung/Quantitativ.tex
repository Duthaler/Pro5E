\subsection{Quantitative Versuche}\label{subsec:Quantitative}
In diesem Versuch werden kurz die Versuche und deren Erkenntnis erläutert, bei welchen die Ergebnisse nicht qualitativ ermittelt werden konnten.
\begin{itemize}
	\item \textbf{Welligkeit:} Da die Welligkeit der Zugkraft maximal $\pm$25N betragen darf (Anhang \ref{appsec:Anforderungen} Punkt 01-03), wurde das Verhalten bei Änderung der Drehzahl und der Leistung beobachtet. Dabei konnten keine Kraftschwankungen auf dem System wahrgenommen oder gemessen werden.
	\item \textbf{Lastabwurf:} Um festzustellen, wie das System bei einem plötzlichen Lastabwurf reagiert, wurde dieser ebenfalls simuliert. Dabei hat sich gezeigt, dass die Drehzahl kurzzeitig stark ansteigt (auch über die Drehzahlbegrenzung), die Ansteuerung danach jedoch innerhalb von wenigen Sekunden auf die Drehzahl des Maximums zurückregelt.
	\item \textbf{Geschwindigkeitsbegrenzung: } Über die Software der Ansteuerung lässt sich auf dieser eine maximal Geschwindigkeit einstellen. Beim erreichen der Geschwindigkeitsbegrenzung hat sich gezeigt, dass die Ansteuerung innerhalb ein paar Sekunden die Kraft und somit auch die Geschwindigkeit reduziert. Die Zeit, welche die Ansteuerung braucht bis diese die Kraft reduziert und das Überschwingen der Geschwindigkeit ist abhängig von der Last und der Beschleunigung des Motors. Beides ist jedoch innerhalb eines tolerierbaren Bereiches.
	\item \textbf{Geschwindigkeitsregelung:} Da der BLDC-Motor eine Kraftregelung und durch den Permanentmagnet ein grosses Haltemoment hat, ist eine Geschwindigkeitsreglung im Leerlauf bei niedrigen Drehzahlen (U<800 RPM) schwierig. Dies aufgrund einer zeitlichen Verzögerung und einem hohen Übertragungswert zwischen Eingang- und Ausgangssignal der Ansteuerung. Bei höheren Drehzahlen oder unter Belastung ist die Geschwindigkeitsreglung einfacher, da durch die Kraftregelung der Übertragungswert kleiner wird.
\end{itemize}