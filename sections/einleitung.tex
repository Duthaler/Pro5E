\section{Einleitung}
Vor über 50 Jahren stürzten sich mit den ersten Gleitschirmen waghalsige den Hang herunter. Die Technologie machte wahnsinnige Fortschritte, jedoch blieb das Startgelände lange den Bergen verwahrt. Obwohl die Schweiz über hohe Berge verfügt und dementsprechend viele Gleitschirmpiloten von Bergen her starten, gibt es gerade im nördlichen Teil der Schweiz, aufgrund von niedrigeren und bewaldeten Bergen, praktisch keine Startmöglichkeiten. Um dem entgegenzuwirken wurden sogenannte Abroll- und Einzugswinden entwickelt, welche einen Start in flachem Gelände ermöglichen. Da diese Technik bereits länger existiert, und die meisten Varianten auf einer Lösung mit Verbrennungsmotoren beruht, sind elektrisch betriebene Winden bisher kaum vertreten. Durch den technischen Fortschritt von Akkumulatoren ist ein regelrechter Aufschwung im Bereich von elektrischen Anwendungen im Gange. Mit der Kombination eines Elektromotors sind diverse Vorteile gegeben wie beispielsweise ein geräuscharmer Betrieb, keine CO2 Emissionen, einem hohen Wirkungsgrad und einer bidirektionalen Anwendung. Letzterer Punkt ermöglicht den Betrieb so wohl in Vorwärtsrichtung (als Motor), als auch in Rückwärtsrichtung (als Generator). Dies ermöglicht für den Abrollwinden-, und den Einzugswindenbetrieb die selbe Winde zu verwenden, was bis zum jetzigen Zeitpunkt absolut neuartig ist. Genau für diesen Zweck wird die Dualwinch entwickelt. Unser Ziel ist es, das Gleitschirmfliegen im Flachland zu revolutionieren und dabei die Umwelt nicht ausser Acht zu lassen. Dabei ist eine einfache und sichere Bedienung für jedermann und einem effektiven Schleppvorgang unsere Motivation.


In diesem Bericht werden einige Begrifflichkeiten verwendet, welche nachfolgend definiert und kurz beschrieben werden.
\begin{table}[H]
	\centering
	\begin{tabular}{p{3cm} p{11cm}}
		\multicolumn{2}{c}{}\\
		{Begriff} & {Erklärung} \\ \hline
		Einzugwinde oder Einzugbetrieb    &   Hier wird der Pilot zur Winde, welche stationär am Boden steht, hingezogen und das Seil dementsprechend eingezogen (aufgerollt). Die Startentfernung zwischen Pilot und Winde beträgt einige hundert Meter.   \\
		Abrollwinde oder Abrollbetrieb    &    Hier ist ein Auto die treibende Kraft und bewegt sich während dem Betrieb vom Piloten weg. Die Winde, welche auf dem Auto befestigt ist, beginnt beim Abrollvorgang zu bremsen, so dass eine Kraft am Piloten entsteht und dieser in die Luft gehoben wird. Die Startentfernung zwischen Pilot und Winde (resp. Auto) beträgt lediglich einige Meter.  \\
		ASM    &  Ist die Kurzschreibeweise für Asynochronmotor und stellt die Lastmaschine bei den Laborversuchen dar. \\
		BLDC    &  Ist die Kurzschreibweise für Brushless DC (Gleichstrom) Motor und beschreibt den für die Winde erworbenen Motor. \\
		RPM   & Bedeutet ausgeschrieben Rounds per Minute und zu Deutsch nichts anderes als Umdrehungen pro Minute. Wird für die Drehzahl bei Motoren verwendet 
	\end{tabular}
	\caption{Begrifferläuterungen}
	\label{tab:Begrifferlaeuterungen}
\end{table}


