\section{Einleitung}
Vor über 50 Jahren stürzten Waghalsige mit den ersten Gleitschirmen den Hang hinunter. Die Technologie machte etliche Fortschritte, doch musste stets an einem Berghang gestartet werden. Obwohl die Schweiz über hohe Berge verfügt und dementsprechend viele Gleitschirmpiloten hat, gibt es gerade im nördlichen Teil der Schweiz, aufgrund niedriger und bewaldeter Hügel, praktisch keine Startmöglichkeiten. Um dem entgegenzuwirken, wurden sogenannte Abroll- und Einzugswinden entwickelt, welche einen Start in flachem Gelände ermöglichen. Da diese Technik bereits länger existiert und die meisten Varianten auf einer Lösung mit Verbrennungsmotoren beruht, gibt es kaum elektrisch betriebene Winden. Dank des technischen Fortschritts von Akkumulatoren sind in praktisch jedem Bereich elektrische Anwendungen möglich. Eine Schleppwinde mit Elektromotor und elektrischer Energieversorgung hat diverse Vorteile wie beispielsweise ein geräuscharmer Betrieb, keine CO$_{2}$ Emissionen, einen hohen Wirkungsgrad und eine bidirektionalen Anwendung. Letzterer Punkt ermöglicht den Betrieb sowohl in Vorwärtsrichtung (als Motor) als auch in Rückwärtsrichtung (als Generator). Dies erlaubt, für den Abrollwinden- und Einzugswindenbetrieb dieselbe Winde zu verwenden, was bis zum jetzigen Zeitpunkt absolut neuartig ist. 
Ziel der vorliegenden Arbeit ist es, das Gleitschirmfliegen im Flachland zu revolutionieren und dabei die Umwelt nicht ausser Acht zu lassen. Genau für diesen Zweck wird in diesem Projekt eine ebensolche elektrische Winde entwickelt und in Form eines Prototypen realisiert und getestet.

Damit am Ende des Projektes ein reibungsloser Windenbetrieb garantiert werden kann, sind Anforderungen an das gesamte System notwendig. Da der Windenschlepp in der Schweiz nicht sehr verbreitet ist, sind verglichen mit Deutschland praktisch keine Vorschriften vorhanden. Aus diesem Grund sind die Anforderungen unserer Winde an die Richtlinien des DHV (Deutscher Hängegleiterverband) angelehnt.
Damit der Pilot möglichst sicher in die Luft gelangt, ist neben einer maximalen Einzugsgeschwindigkeit des Seils auch eine minimale und maximale Zugkraft am Seil notwendig, welche abhängig vom Körpergewicht des Piloten vorgängig eingestellt werden kann. Da der Schleppvorgang dynamisch ist und sich die Zugkraft am Seil je nach Luftschicht ändern kann, ist eine Regelung der Zugkraft zwingend notwendig. Weiter gilt es die Festigkeit aller mechanischen Komponenten zu beachten und Elemente wie Kappvorrichtung und Sollbruchstellen zu berücksichtigen. Die gesamte Energieversorgung des Systems erfolgt elektrisch und beruht primär auf Batteriebetrieb. Sollte der Energiespeicher erschöpft sein, dient ein externer Generator als Lade- und Stützeinheit der Batterie und übernimmt lediglich einen Teil der notwendigen Leistung.

In diesem Projekt wurden zuerst die rechtlichen Rahmenbedingungen abgeklärt, damit darauf basierend die Anforderungen an das Produkt eruiert werden konnten. Ein Pflichtenheft \cite{TechPflichtenheft}, welches die Anforderungen (Anhang \ref{appsec:Anforderungen}) an das Produkt definiert, wurde in enger Zusammenarbeit mit dem Projektteam des mechanischen Teils erarbeitet. Nachdem die Anforderungen an den Motor spezifiziert werden konnten, wurde dieser beschafft und auf \glqq Herz und Niere\grqq\space getestet.

Dieser Bericht ist in vier Themengebiete unterteilt, wobei anfänglich die Grundlagen zum Windenschlepp und dessen genauen Richtlinien erläutert werden. Im Abschnitt Hardware werden alle für den Prototypen notwendigen Komponenten aufgelistet und detailliert auf deren Funktion beschrieben. Die durchgeführten Tests von Motor und Controller und dessen Auswertung sind im Abschnitt Validierung ersichtlich. Einen Einblick in die Zukunft gewährt der Abschnitt Offene Arbeiten. Dieser fasst die nächsten Schritte zusammen, welche notwendig sind um das Projekt abzuschliessen.

