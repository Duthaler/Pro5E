\section{Einleitung}
Vor über 50 Jahren stürzten sich mit den ersten Gleitschirmen waghalsige den Hang herunter. Die Technologie machte wahnsinnige Fortschritte, jedoch blieb das Startgelände lange den Bergen verwahrt. Obwohl die Schweiz über hohe Berge verfügt und dementsprechend viele Gleitschirmflieger von Bergen her starten, gibt es gerade im nördlichen Teil der Schweiz, aufgrund von niedrigeren und bewaldeten Bergen, praktisch keine Startmöglichkeiten. Um dem entgegenzuwirken wurden sogenannte Schlepp- und Einzugswinden entwickelt, welche einen Start in flachem Gelände ermöglichen. Da diese Technik bereits länger existiert, und die günstigste Variante auf einer Lösung mit Verbrennungsmotor beruht, sind elektrische Winden bisher kaum vertreten. Durch den technischen Fortschritt von Akkumulatoren ist ein regelrechter Aufschwung im Bereich von elektrischen Anwendungen im Gange. Neben einem geräuscharmen Betrieb, keinen CO2 Emissionen und einem hohen Wirkungsgrad, weisen gerade Elektromotoren diverse Vorteile auf. Mit der bidirektionalen Anwendung kann der Motor in Vorwärtsrichtung (als Motor) und in Rückwärtsrichtung (als Generator) betrieben werden, und somit sowohl als Schlepp- als auch als Einzugswinde betrieben werden kann.