\begin{abstract}
For a long time, not only the weather but also the terrain determined whether a paraglider could be flown. Towing winches can practically eliminate the terrain factor, but in order to cover all eventualities, both pull-in winches and roll-off winches still have to be purchased as separate systems. To solve this problem, the DualModeWinch is being developed in this project. It has both the pull-in winch and the roll-off winch in one system. It is the first winch on the market that combines both functions in one device. This is made possible by the installation of an electric motor which can be operated both in forward and reverse direction. In addition, the power supply is provided by batteries which when compared to conventional towing winches enables low-noise, no C0$_{2}$ emissions and an energy-efficient operation. The motor is monitored by a controller during the towing process. It also ensures that the operation is as error-free and safe as possible. The DualModeWinch is delivered as a complete package and can be transported in a car with boot size of 90x60x90cm. It is therefore suitable for private individuals as well as for larger paragliding clubs. The central elements of the winch are the motor and the associated controller, whereby the controller regulates the motor in both operating modes. The first performance tests in pull-in winch operation were successful. All legal requirements for the winch as well as all possible operating conditions for torque and speed were achieved. In a future step, all important components for the discharge circuit, the heating resistor and the batteries must be determined, ordered and assembled for the test setup. In summary, it can be said that all general conditions and legal requirements have been met and a large part of the functionality of the winch could be tested.


\vspace{2ex}
\textbf{Keywords: Dual mode, Electric, Paraglide, Winch}
\end{abstract}	



