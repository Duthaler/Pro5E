\section{Schlusswort}\label{sec:Schlusswort}

In dieser Arbeit wurde versucht, alle elektrisch betriebenen Bestandteile einer Schleppwinde in Abrollwinden- und Einzugswindenbetrieb zu definieren und mittels Laborversuchen auszutesten. Dabei galt es alle Anforderungen an eine Schleppwinde (gemäss dem deutschen Hängegleiterverbande) zu beachten und dementsprechend alle Bestandteile nach den geforderten Normen auszulegen.

Die Ergebnisse der Labortests zeigen, dass der Motor alle Anforderungen im Einzugswindenbetrieb erfüllt. Die Ansteuerung des Motors erfolgt mittels eines Controllers und verläuft ebenfalls optimal. Hierbei kann eine maximale Drehzahl vorgegeben werden, welche vom Motor bei Normalbetrieb nicht überschritten werden kann. Dadurch lässt garantieren, dass eine Geschwindigkeit des Seileinzugs von 10m/s nicht überschritten wird. Auch bei Lastabwurf, regelt der Controller extrem schnell und reduziert die Geschwindigkeit des Motors auf die gewünschte Drehzahl herunter. Weiter können Drehmomentsollwerte über den Controller sehr einfach vorgegeben und vom Motor angefahren werden. Weiter kann trotz Leistungsänderungen während dem Betrieb keine Welligkeit festgestellt werden. Erfreulich ist auch das Leistungsverhalten bei Änderungen der Temperatur mit einer Abweichung von lediglich 6\% über den ganzen Temperaturbereich (30°C zu 100°C). Auch der Betrieb in Feldschwächung funktioniert einwandfrei, wobei beachtet werden muss, dass diese von der Spannungsversorgung abhängig ist. Unterschreitet die Spannungsversorgung 84V, so kann die Drehzahl von 3800U/min (entspricht Drehzahl im Nennarbeitspunkt) auch im Leerlauf nicht mehr erreicht werden. Wird der Motor unterhalb der Nennspannung (96V) betrieben, kann er das Nennmoment bei hohen Drehzahlen nicht mehr erbringen, dadurch ergibt sich eine Leistungsgrenze, welche von der Versorgungsspannung abhängig ist. Damit diese so klein wie möglich ist, darf die Energieversorgung auch unter starker Belastung keinen Markanten Spannungseinbruch aufweisen.

Leider konnte in diesem Projekt mangels Zeit der Abrollwindenbetrieb nicht ausgearbeitet werden. Aus diesem Grund soll als nächstes genau dieser ausgelegt und getestet werden. Dafür sind sowohl Batterien als auch Heizwiderstände und die dazugehörige Elektronik notwendig. Mit diesen Komponenten wird der Versuchsaufbau erweitert, damit der Motor sowohl im Einzugswindenbetrieb als auch Abrollwindenbetrieb betrieben und validiert werden kann. Während der mechanische Teil konstruiert und aufgebaut wird, wird das Bedieninterface und die übrigen Bauteile entwickelt, damit am Anfang des nächsten Sommers die ersten Feldversuche möglich sind.

