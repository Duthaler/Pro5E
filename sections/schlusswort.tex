\section{Schlusswort}\label{sec:Schlusswort}
Am Anfang des Projekts, bevor die Anforderungen an den Motor gestellt werden konnten, mussten zuerst die rechtlichen Vorschriften für Gleitschirmwinden abgeklärt werden. Erst nachdem diese geklärt wurden, konnten Anforderungen an die ersten Bauteile eruiert werden. Im zweiten Teil des Projekts hingegen wurde der Motor intensiv auf seine Anforderungen getestet.\\
Das Ziel des Projekts war vor allem neben den gesetzlichen Rahmenbedingungen die Validierung des Motors. Dies ist von Nöten, damit die Planung des mechanischen Teils fortgeführt werden kann.\\
Anhand der Validierung des Motors konnte bewiesen werden, dass dieser alle Anforderungen für den Einzugsbetrieb erfüllt.\\
Als nächstes sollten sowohl die Batterien als auch die Heizwiderstände und die dazugehörige Elektronik dimensioniert und beschafft werden. Mit diesen Komponenten wird der Versuchsaufbau erweitert, damit der Motor auch für den Abrollbetrieb validiert werden kann. Während der mechanische Teil konstruiert und aufgebaut wird, wird das Bedieninterface und die übrigen Bauteile entwickelt, damit am Anfang des nächsten Sommers die ersten Feldversuche möglich sind.



(Schlusswort ist noch nicht fertig!)
